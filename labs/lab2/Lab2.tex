\documentclass[twocolumn]{article}
\usepackage[pdftex]{graphicx}


\title{Lab 2: Electric Fields \& Electric Potential}
\author{Jason Soo}

\begin{document}
\maketitle
\section{Introduction} % (fold)
\label{sec:introduction}

In physics, the space surrounding an electric charge or in the presence of a time-varying magnetic field has a property called an electric field (that can also be equated to electric flux density). This electric field exerts a force on other electrically charged objects. The concept of electric field was introduced by Michael Faraday.

The electric field is a vector field with SI units of newtons per coulomb (N C$^-$$^1$) or, equivalently, volts per meter (V m$^-$$^1$). The strength of the field at a given point is defined as the force that would be exerted on a positive test charge of +1 coulomb placed at that point; the direction of the field is given by the direction of that force. Electric fields contain electrical energy with energy density proportional to the square of the field intensity. The electric field is to charge as gravitational acceleration is to mass and force density is to volume.

A moving charge has not just an electric field but also a magnetic field, and in general the electric and magnetic fields are not completely separate phenomena; what one observer perceives as an electric field, another observer in a different frame of reference perceives as a mixture of electric and magnetic fields. For this reason, one speaks of ``electromagnetism'' or ``electromagnetic fields.'' In quantum mechanics, disturbances in the electromagnetic fields are called photons, and the energy of photons is quantized.

At a point in space, the electric potential is the potential energy per unit of charge that is associated with a static (time-invariant) electric field. It is typically measured in volts, and is a Lorentz scalar quantity. The difference in electrical potential between two points is known as voltage.

There is also a generalized electric scalar potential that is used in electrodynamics when time-varying electromagnetic fields are present. This generalized electric potential cannot be simply interpreted as potential energy per unit charge, however.
% section introduction (end)

\section{Theory} % (fold)
\label{sec:theory}
We will obtain visual clarity for density of electric field line density is proportional to the electric field strength.  By mapping equipotential lines we can confirm the above statement.  We expect that the density will be greatest around the positive point charge -- where the charge is greatest.

% section theory (end)


\section{Equipment List} % (fold)
\label{sec:equipment_list}
\begin{itemize}
\item[*] Digital Multimeter
\item[*] Masses
\item[*] Metal Push Pins
\item[*] Conductive Paper
\item[*] Alligator Clips
\end{itemize}
% section equipment_list (end)



\section{Procedure}\label{sec:procedure}
	\subsection{Part A}\label{sub:part_a}

	Place the conductive paper onto a cork board.  Secure it with the metal push pins.  
	By doing this, we create two potential points charges.  We connected the power supply to 
	one point charge.  We then used the DMM to trace the equal potential lines of various 
	charges.

	\subsection{Part B} % (fold)
	\label{sub:part_b}

	In this section we simply adjust the location of the point charges and place conductive ink on the conductive paper to see its affects.
	% subsection part_b (end)


\section{Data} % (fold)
\label{sec:data}

\subsection{Part A} % (fold)
\label{sub:part_a}
	
	By calculation the magnitude of the electric field at various points we can validate our theory.  See Table 1 for results.  The values were obtained using Equation 1.

	\begin{equation}
		|E| = |\frac{\Delta V}{\Delta d}|
	\end{equation}

	\begin{table}[b]
		\begin{center}

		\begin{tabular}{ccc}
		\textbf{$\Delta V$} & \textbf{$\Delta d$} & \textbf{$|E|$}\\
		\hline
		7 & 1 & 7\\\\
		5 & 4 & $\frac{5}{4}$\\\\
		4 & 6 & $\frac{2}{3}$
		\end{tabular}
		\label{tab:part_1_mag}
		\caption{Electric Field Strength}

		\end{center}
	\end{table}

	By viewing the data shown is table one it is clear that as we move further away from the positive pole that we see a decrease in the density of equipotential lines as the electric fields begin to dissipate.  They do however begin again to gather around the negative point charge because of the charge configuration of the electric dipole system.

	% subsection part_a (end)
	
	\subsection{Part B} % (fold)
	\label{sub:part_b}
	
	See attached graph.
	
	% subsection part_b (end)

% section data (end)


\section{Analysis of Data}\label{sec:analysis_of_data}

\subsection{Part A}\label{sub:part_a}
The electric fields are pointing away from the positive pole (the one which has a greater density of equipotential lines).  They point inward to the point charge with 0 volts, the negative one in this case.

\subsection{Part B}\label{sub:part_b}
The direction of the electric field is perpendicular to the 
equipotential lines since its component along the equipotentials is zero (V 
is a constant). We can therefore immediately trace the path of any test particle in the 
field, since charges move in the direction of least energy. The force it experiences is 
equal to the product of its charge and the electric field it is in, so the particle moves 
along the gradient, perpendicular to the equipotentials.

We can observe the strength of the electric field based on the radius and orientation of the circles with respect to a point charge.  When the equipotential lines are closely oriented around a point charge, the electric field has a strong charge.

If we doubled the power supply the equipotential lines would likely double their radius.  Thus, the strength of the electric field would expand further as there would be double the number of equipotential geometric shapes formed.

\section{Conclusions}\label{sec:conclusions}
By visualizing the equipotential lines at various $\Delta V$ it gives us the opportunity to graphically prove the theory that the density of equipotential lines is proportional to the magnitude of the electric field ($|E|$).

\end{document}
\documentclass[twocolumn]{article}

\newenvironment{small_item}{
\begin{itemize}
  \setlength{\itemsep}{.25pt}
  \setlength{\parskip}{0pt}
  \setlength{\parsep}{0pt}
}{\end{itemize}}

\usepackage{times}
\usepackage[small,compact]{titlesec}
\usepackage{simplemargins}
\setallmargins{.25in}

\begin{document}
	
	\section*{Chapter 22} % (fold)
	\label{sec:chapter_22}
	
	\subsection*{22.1} % (fold)
	\label{sub:22_1}
	\begin{small_item}
		\item $\mu = 10^{-6}$
		\item $\epsilon_0 = 8.85*10^{-12}\frac{C^2}{Nm^2}$
		\item $M_e = 9.109 * 10^{-31}kg$
		\item $M_p = 1.672 * 10^{-27}kg$
		\item $F_g = 6.67 * 10^{-11}\frac{Nm^2}{kg^2}$
		\item When you connect a $Q^+$ to the ground i becomes neutralized.
		\item SI unit of electric charge: \textbf{Coulombs (C)}
		\item Charge of electron/proton: $\pm1.60 X 10^{-19}C$
		\item \textbf{Ions:} Atoms with a missing/extra electron.
	\end{small_item}
	% subsection 22_1 (end)
	
	\subsection*{22.2 - Coulomb's Law} % (fold)
	\label{sub:22_2}
	
	\begin{small_item}
		\item $q'$ exerts the force on $q$.
		\item \textbf{Coulomb's Law} is use to find electric force (F) that a particle of charge $q'$ exerts on a particle of charge $q$ at distance $r$.
		\item $\mathbf{|F|} = \frac{1}{4\pi\epsilon_{0}}\frac{q'q}{r^2}$ where $\frac{1}{4\pi\epsilon_{0}} = 8.99 X 10^9 N*m^2/C^2$
		\item Repulsive force if $F > 0$, attractive if $F < 0$.
		\item Coulomb's law applies to particles (electrons and protons) and small charged bodies, that is, where the distance between them is much great than the charge.  These are called point charges.
	\end{small_item}
	% subsection 22_2 (end)
	
	\subsection*{22.3 - The Superposition of Electric Forces} % (fold)
	\label{sub:22_3_the_superposition_of_electric_forces}
	
	\begin{small_item}
		\item The electric force is a vector.
		\item Superposition Principle of Electric Forces: $\mathbf{F = F_{1} + F_{2} + ...}$
		\item Look for symmetry to make calculations easier.  Symmetry generally leads to force components canceling thus easier vector sums. (Figure 22.8/Example 22.6)
		\item \emph{Note:} The $y$ component of $\mathbf{F}_1$ is $-F_1\,cos\,\theta$.  Not always true.  \textbf{How is this determined??!!??}
	\end{small_item}
	% subsection 22_3_the_superposition_of_electric_forces (end)
	
	\subsection*{22.4 - Charge Quantization and Conservation} % (fold)
	\label{sub:22_4_charge_quantization_and_conservation}
	
	\begin{small_item}
		\item Antiparticles have opposite charges.
		\item All charge is discrete (quantized).  ie., some multiple of the fundamental charge $e$.  For ease we treat large bodies as continuos.
	\end{small_item}
	% subsection 22_4_charge_quantization_and_conservation (end)
	
	\subsection*{22.5 - Conductors and Insulators; Charging by Friction or by Induction} % (fold)
	\label{sub:22_5_conductors_and_insulators_charging_by_friction_or_by_induction}
	
	\begin{small_item}
		\item \textbf{Conductor:} a material that permits the motion of electric charges through its volume.  Charges immediately spread out and reach an equilibrium. 
		\item \textbf{Insulator:} is a material that does not readily permit the motion of electric charges.  Charges to not move when they come in contact with an insulator.
		\item A body will acquire a net positive charge if electrons are removed and a net negative charge if electrons are added.
		\item \textbf{Electrolytes:} Liquid conductors with an abundance of ions.
	\end{small_item}
	% subsection 22_5_conductors_and_insulators_charging_by_friction_or_by_induction (end)
	
	% section chapter_22 (end)
	
	\section*{Chapter 23 - The Electric Field} % (fold)
	\label{sec:chapter_23_the_electric_field}
	
	\subsection*{23.1 - The Electric Field of Point Charges} % (fold)
	\label{sub:23_1_the_electric_field_of_point_charges}
	
	\begin{small_item}
		\item \textbf{Electric Field:} Charges exert forces on one another by means of disturbances that they generate in the space surrounding them via disturbances known as electric fields.
		\item Electric force cannot move faster than the speed of light.
		\item Electric field of a point charge: $E = \frac{1}{4\pi\epsilon_0}\frac{q'}{r^2}$
		\item The force that the electric field exerts on q is $\mathbf{F} = q\mathbf{E} = M\mathbf{a}$
		\item The electric field \& electric force: $\mathbf{E} = \frac{\mathbf{F}}{q}N/C$
		\item SI unit for $\mathbf{E}$ is $\frac{Newtons}{Coulimb} = \frac{Volts}{Meter}$
		\item The net electric field generated by any distribution of point charges can be calculated by forming the vector sum of the individual electric fields due to the point charges.
		\item If $-\mathbf{E}$ that means acceleration is in opposite direction of the field.
	\end{small_item}
	% subsection 23_1_the_electric_field_of_point_charges (end)
	
	\subsection*{23.2 The Electric Field of Continuos Charge Distributions} % (fold)
	\label{sub:23_2_the_electric_field_of_continous_charge_distributions}
	
	\begin{small_item}
		\item Magnitude of electric field contribution: \\$dE = \frac{1}{4\pi\epsilon_0}\frac{dq}{r^2}$ where $dq$ is a small region within a continuos body that contributes a field of magnitude $dE$.  Assuming the region is small enough it is treated as a point charge.
		\item x component of electric field contribution: $dE_x = \frac{1}{4\pi\epsilon_0}\,\frac{cos\,\theta}{r^2}\,dq$
		\item Ugh...should probably re-read this section...
		\item $E = \frac{1}{2\pi\epsilon_0}\frac{\lambda}{r}$ - Used for finding charge on extremely long rods..;). $\lambda$ is the charge per unit length.
		\item Electric field of a flat sheet: $E_x = \frac{\rho}{2\epsilon_o}$.  Used for finding charge on extremely long disks.
	\end{small_item}
	% subsection 23_2_the_electric_field_of_continous_charge_distributions (end)
	
	\subsection*{23.3 Lines of Electric Field} % (fold)
	\label{sub:23_3_lines_of_electric_field}
	
	\begin{small_item}
		\item The density of the field lines represents the strength of the electric field.
		\item Two times as many field lines originate from a $2q$ charge than a $q$ charge.
		\item Field lines never intersect.  Thus fields can have only one direction.
		\item At distance $r$ from the point charge these lines are uniformly distributed over the area $A = 4\pi r^2$
	\end{small_item}
	% subsection 23_3_lines_of_electric_field (end)
	
	
	\subsection*{23.4 Motion in a Uniform Electric Field} % (fold)
	\label{sub:23_4_motion_in_a_uniform_electric_field}

	\begin{small_item}
		\item Velocity for x and y components:\\
		$v_x = v_{0x} + a_xt$ and $v_y = v_{0y} + a_yt$ where $v_{0x}$ and $v_{0y}$ are initial velocity components.  $a_x = \frac{F_x}{m} = \frac{qE_x}{m}$ and $a_y = \frac{F_y}{m} = \frac{qE_y}{m}$.
		\item Position eq: $x = x_0 + v_{0x}t + \frac{1}{2}a_xt^2$
	\end{small_item}	
	% subsection 23_4_motion_in_a_uniform_electric_field (end)	
	

	\subsection*{23.5 Electric Dipole in an Electric Field} % (fold)
	\label{sub:23_5_electric_dipole_in_an_electric_field}
	
	\begin{small_item}
		\item \textbf{External Force:} The electric field that acts on a body.
		\item \textbf{Self-field:} The field generated by the body itself.  This field exerts internal forces within the body and does not contribute to the net force acting on the body from outside.
		\item A body may experience a torque in a uniform external electric field.
		\item Torque on electric dipole: $\tau = -pE\,sin\,\theta$, where $p = lQ$.  $p$ is known as the \textbf{dipole moment} of the body.
		\item Work done to rotate the dipole to some other angle $\theta$: $W = pE\,cos\,\theta$.
		\item The potential is the negative of the work done: $U = -W = -pE\,cos\,\theta$
		\item Stopped at equation 23.20.
	\end{small_item}
	% subsection 23_5_electric_dipole_in_an_electric_field (end)

	% section chapter_23_the_electric_field (end)
	
	
	\section*{Chapter 24 Gauss' Law} % (fold)
	\label{sec:chapter_24_gauss_law}
	
	\subsection*{24.1 Electric Flux} % (fold)
	\label{sub:24_1_electric_flux}
	
	\begin{small_item}
		\item \textbf{Electric flux:} $\Phi_E$ through the surface is defined as the product of the area A and the normal component of the electric field.
		\item $\Phi_E = EA\,cos\,\theta = E_\perp A$, where $E_\perp$ is the perpendicular component.
		\item SI unit for electric flux is $N*m^2/C$
		\item $E_\perp$ can be expressed as $E\,cos\,\theta$ where $\theta$ is the angle between $\mathbf{E}$ and $\perp$ to the surface.
		\item If $\theta = 0^\circ$ (ie. the surface is $\perp$ to the electric field thus intercepting maximum field lines) then $\Phi_E = EA$.
		\item If $\theta = 90^\circ$ (ie. the surface is $||$ to the electric field thus intercepting no field lines) then $\Phi_E = 0$.
		\item The normal component $E_\perp$ is reckoned as positive if the direction of the electric field \textbf{E} is outward from the surface, and negative if \textbf{E} is inwards, into the surface.
		\item $\Phi = \mathbf{E}\bullet \mathbf{A}$
		\item To find charge density of faces/surfaces: $\rho = E \epsilon_0$
		\item Total charge of plates: $Q = A \rho$
	\end{small_item}
	% subsection 24_1_electric_flux (end)
	
	
	\subsection*{24.2 Gauss' Law} % (fold)
	\label{sub:24_2_gauss_law}
	
	\begin{small_item}
		\item \textbf{Gauss' Law:} If an arbitrary closed surface has a net electric charge $Q_{inside}$ within it, then the electric flux through the surface is $\frac{Q_{inside}}{\epsilon_0}$.
		\item Gauss' Law: $\Phi_E = \oint E_\perp dA = \frac{Q_{inside}}{\epsilon_0}$
	\end{small_item}
	% subsection 24_2_gauss_law (end)
	
	
	\subsection*{24.3 Applications of Gauss' Law} % (fold)
	\label{sub:24_3_applications_of_gauss_law}
	
	\begin{small_item}
		\item Line: $q = \lambda L$, where $\lambda$ is the charge per unit length in coulombs per meter $C/m$.
		\item Surface: $q = \sigma A$, where $\sigma$ is the charge per unit area in coulombs per square meter, $C/m^2$.
		\item Volume: $q = \rho V$, where $\rho$ is the charge per unit volume is coulombs per cubic meter, $C/m^3$
	\end{small_item}
	% subsection 24_3_applications_of_gauss_law (end)
	
	% section chapter_24_gauss_law (end)
	
	
	
	\section*{Chapter 25 Electrostatic Potential and Energy} % (fold)
	\label{sec:chapter_25_electrostatic_potential_and_energy}
	
	\subsection*{25.1 The Electrostatic Potential} % (fold)
	\label{sub:25_1_the_electrostatic_potential}
	
	\begin{small_item}
		\item Work $= U_1 - U_2 = \frac{qq'}{4\pi\epsilon_0}(\frac{1}{r_1}
		-\frac{1}{r_2})$
		\item Work done depends only on radial distance.
		\item $K + U = \frac{1}{2}mv^2 - qE_0y$
		\item \textbf{Electric Potential Energy:} $U = \frac{1}{4\pi\epsilon_0}\frac{qq'}{r}$
		\item \textbf{Electrostatic Potential:} $V = \frac{U}{q}$
		\item SI Unit: 1 volt = 1 $\frac{joule}{coulomb}$
		\item For two charges of equal signs, the electric potential energy is positive and it decreases in inverse proportion to the distance.
		\item For opposite signs, its negative and decreases in the same manor.
		\item \textbf{Coulomb potential:} $V = \frac{1}{4\pi\epsilon_0}\frac{q'}{r}$.  Used to find the electrostatic potential of a point charge.
		\item Avogadro’s Number: $-N_A = 6.02*10^{23}$
		\item $\Delta V = \frac{Work}{Q}$
	\end{small_item}
	% subsection 25_1_the_electrostatic_potential (end)
	
	\subsection*{25.2 Calculation of the Potential from the Field} % (fold)
	\label{sub:25_2_calculation_of_the_potential_from_the_field}
	\begin{small_item}
		\item Electrostatic potential from field: $V = -\int_{P_0}^P E\,cos\,\theta\,ds+V_0$
	\end{small_item}
	% subsection 25_2_calculation_of_the_potential_from_the_field (end)
	
	\subsection*{25.3 Potential in Conductors} % (fold)
	\label{sub:25_3_potential_in_conductors}
	
	\begin{small_item}
		\item All points within a conducting body are at the same electrostatic potential.
		\item Potential of the Earth's surface is $V = 0$, zero.
		\item Outside a sphere the potential is the same as a point charge,\\ $V = \frac{1}{4\pi\epsilon_0}\frac{Q}{r}\: (r \ge R)$
		\item Inside the conductor, the potential is the same everywhere, thus $V = \frac{1}{4\pi\epsilon_0}\frac{Q}{R}\: (r \le R)$
	\end{small_item}
	% subsection 25_3_potential_in_conductors (end)

	\subsection*{25.4 Calculation of the Field from the Potential} % (fold)
	\label{sub:25_4_calculation_of_the_field_from_the_potential}
	
	\begin{small_item}
		\item Electric field from potential: $E = \frac{dV}{ds}$, where $V - V_0 = dV$ and $ds$ is a small displacement in the direction of the electric field.
		\item Components of the electric field: $E_x = -\frac{\partial V}{\partial x}$, etc.
	\end{small_item}
	% subsection 25_4_calculation_of_the_field_from_the_potential (end)
	
	\subsection*{25.5 Energy of Systems of Charges} % (fold)
	\label{sub:25_5_energy_of_systems_of_charges}
	
	\begin{small_item}
		\item Potential energy of a system of point charges:\\ $U = \frac{1}{4\pi\epsilon_0}\frac{q_1q_2}{r_{12}} + \frac{1}{4\pi\epsilon_0}\frac{q_2q_3}{r_{23}} + \frac{1}{4\pi\epsilon_0}\frac{q_1q_3}{r_{13}}$
		\item Potential energy of a Conductor: $U = \frac{1}{2}QV$
		\item For a system of conductors, just sum the various potentials.
		\item Electric potential energy for two charged plates separated by distance $d$: $U = \frac{1}{2}\frac{Q^2d}{\epsilon_0A} = \frac{1}{2}\epsilon_0(\frac{Q}{\epsilon_0A})^2Ad = \frac{1}{2}\epsilon_0E^2*Ad$
		\item For a volume: $U = \frac{1}{2}\epsilon_0E^2 * [volume]$
		\item Energy density in electric field: $u = \frac{1}{2}\epsilon_0E^2$
	\end{small_item}
	% subsection 25_5_energy_of_systems_of_charges (end)
	
	
	% section chapter_25_electrostatic_potential_and_energy (end)
	
	
	\section*{Chapter 26 Capacitors and Dielectrics} % (fold)
	\label{sec:chapter_26_capacitors_and_dielectrics}
	
	\begin{small_item}
		\item Known: Potential difference and charge density. Want distance of plates of parallel capacitor: $\frac{Q}{A} = \sigma = \frac{\epsilon_0(\Delta V)}{d}$
		\item Known: Capacitors farads and voltage.  Want potential difference is connected in series: $V_n = \frac{Q_n}{C_n} = \frac{C_1C_2V}{C_n(C_1+C_2)}$.  If connected it series its just the voltage since all have the same potential.
		\item ``Over a certain region in space, $V = $ some equation.  Find the electric field in this region.''  Simply take the partial derivatives of each component of the equation with respect to x, y, and z.  Resulting vector is answer.
	\end{small_item}
	
	\subsection*{26.1 Capacitance} % (fold)
	\label{sub:26_1_capacitance}
	
	\begin{small_item}
		\item If the amount of charge placed on a sphere is $Q$ then the potential of the sphere will be $V = \frac{1}{4\pi\epsilon_0}\frac{Q}{R}$
		\item Capacitance of a single conductor: $Q = CV$, where $C$ is the constant of proportionality (the capacitance of the conductor). 
		\item SI unit is farad (F).  $1F = 1\frac{Coulomb}{Volt}$
		\item Capacitance of a pair of conductors: $Q = C\,\Delta V$.
		\item Capacitance of parallel-plate capacitor: $C = \frac{\epsilon_0A}{d}$
	\end{small_item}
	% subsection 26_1_capacitance (end)
	
	\subsection*{26.2 Capacitors in Combination} % (fold)
	\label{sub:26_2_capacitors_in_combination}
	
	\begin{small_item}
		\item Circuit components connected in parallel have the same voltage across each component.
		\item Capacitors in parallel: $C = C_1 + C_2 + ...$
		\item Capacitors in series have the same magnitude of charge on each plate.
		\item Capacitors in series: $\frac{1}{C} = \frac{1}{C_1} + ...$
	\end{small_item}
	% subsection 26_2_capacitors_in_combination (end)
	
	\subsection*{26.3 Dielectrics} % (fold)
	\label{sub:26_3_dielectrics}
	
	\begin{small_item}
		\item The dielectric reduces the strength of the electric field.
		\item Electric field in dielectric: $E = \frac{1}{\kappa}E_{free}$, where $\kappa > 1$ and $E_{free} = \frac{Q}{\epsilon_0A}$
		\item Capacitance of capacitor filled with dielectric: $C = \kappa C_0$, where $C_0 = Q/\Delta V_0$ is the capacitance in the absence of dielectric.
		\item Gauss' Law in dielectrics: $\oint \kappa E_{\perp} dA = \frac{Q_{free, inside}}{\epsilon_0}$
		\item Capacitance per unit length of cylindrical capacitor: $\frac{C_0}{l} = 2\pi\epsilon_0 \frac{1}{ln(b/a)}$ where b and a are radii of the outer and inner conductors.
	\end{small_item}
	% subsection 26_3_dielectrics (end)
	
	
	\subsection*{26.4 Energy in Capacitors} % (fold)
	\label{sub:26_4_energy_in_capacitors}
	
	\begin{small_item}
		\item Potential energy in capacitor: $U = \frac{1}{2}C(\Delta V)^2 = \frac{1}{2}\frac{Q^2}{C}$
		\item Energy density in dielectric: $u = \frac{U}{Ad} = \frac{1}{2}\kappa\epsilon_0E^2$
		\item Good luck.  Remember, its only 10\%.  If you are pissed off, remember that -- and start studying earlier and doing your homework so we can pass this class and not re-take it.  Woo!  You can do it!  (Maybe not right now...but in the long run...you'll get out of this class if you want to.)
	\end{small_item}
	% subsection 26_4_energy_in_capacitors (end)
	
	% section chapter_26_capacitors_and_dielectrics (end)
	
	
\end{document}